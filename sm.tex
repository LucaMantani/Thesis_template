%!TEX root = main.tex

%%%%%%%%%%%%%%%%%%%%%%%%%%%%%%%%%%%%%%%%%%%%%%%%%%%%
%
%      Chapter I :
%
%
%%%%%%%%%%%%%%%%%%%%%%%%%%%%%%%%%%%%%%%%%%%%%%%%%%%

\chapter{The Standard Model of Particle Physics}
\label{chap:sm}
\pagestyle{fancy}

\hfill
\begin{minipage}{10cm}

{\small\it 
``No matter how many instances of white swans we may have observed, this does not justify the conclusion that all swans are white.''}

\hfill {\small Karl Popper, \textit{The Logic of Scientific Discovery}}
\end{minipage}

\vspace{0.5cm}

Over the last century, a combination of extraordinary experimental and theoretical efforts have lead to a deep understanding
of the fundamental structure of the Universe. The result of this endeavour is the formulation of the so-called Standard Model 
of Particle Physics (SM), a theoretical framework that describes the fundamental constituents of Matter and the interactions among them.
The success of the SM can be appreciated for both its remarkable ability to describe every observation to-date and for the 
historical role it had in predicting the existence of particles and interactions that were subsequently confirmed by experiments.
One of the most crucial feature of the SM is the existence of a scalar boson, the Higgs boson, which is necessary to give 
masses to the fundamental particles in a theoretically consistent way. The discovery of a Higgs-like boson at the LHC in 2012~\cite{Aad_2012,Chatrchyan_2012}
provided us with the last piece of the puzzle and paved the way for future discovery and precision programs at CERN and other facilities.
In the following, I will briefly summarise the SM, focusing in particular on the Electroweak (EW) sector and how
gauge symmetries ensure the unitarity of the theory.


\section{The fundamental building blocks of Nature}

The SM is a Quantum Field Theory~\cite{Glashow:1961tr,Weinberg:1967tq,Salam:1968rm} describing three of the four fundamental interactions (electromagnetic, weak and strong force)
and all the known elementary matter constituents. The latter are spin-$\frac{1}{2}$ fermions, classified in two broad categories: \textit{quarks} and 
\textit{leptons}.
The justification to this distinction is given by the fact that while quarks are carriers of the strong interaction charge (called colour), leptons are subjects of the electroweak force only. 
This simple difference has a substantial phenomenological consequence: quarks are not asymptotic states and are confined in colour-neutral bound states called hadrons\footnote{From the greek hadrós which means large, massive.}, such as protons and neutrons. There are six flavours of quarks: up-like quarks (up (u), 
charm (c) and top (t)) carrying electric charge $\frac{2}{3}$ and down-like quarks (down (d), strange(s) and bottom (b)) having electric
charge $-\frac{1}{3}$. These are further pair-related by means of the EW interaction, defining three different generations.
In a similar fashion, leptons are distinguished in charged leptons (electron (e), muon ($\mu$) and tau ($\tau$)) characterised by 
electric charge $-1$ and the corresponding neutrinos $\nu_e$, $\nu_{\mu}$ and $\nu_{\tau}$ with which they form EW pairs.

Interactions among fermionic particles are built upon the fundamental principle of gauge symmetries. The SM is a non-abelian
gauge theory, invariant under the group $G = SU(3)_c \otimes SU(2)_L \otimes U(1)_Y$. Each gauge group introduces in the theory 
massless fundamental gauge bosons (spin-1 particles) in number equal to the dimension of the adjoint representation of the group itself. 
These are the mediating particles of the three fundamental forces. In particular, $SU(3)_c$ describes the strong force by 
means of eight gauge bosons called \textit{gluons} while $SU(2)_L \otimes U(1)_Y$ describes the EW interactions, which are mediated 
by the electroweak bosons $W^\pm$, $Z$ and the photon ($\gamma$).

While photons and gluons are actually massless particles mediating long-range interactions, as dictated by gauge invariance,
the weak bosons $W^\pm$ and $Z$ are responsible for a short-range force and are massive. In addition to this, the SM gauge group prohibit us
to explicitly assign masses to the fermions, which are nonetheless observed in experiments. The solution to this conundrum is
given by the last missing particle in the SM realm, the \textit{Higgs boson}. The mechanism through which the Higgs boson is responsible for
the masses of the aforementioned particles will be illustrated in the following section, in which a more theoretically sound construction of
the EW sector of the SM will be presented.